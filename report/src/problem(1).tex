\section{Условие}
Составить и отладить программу, осуществляющую работу с процессами и взаимодействие между ними в одной из двух операционных систем. В результате работы программа (основной процесс) должен создать для решения задачи один или несколько дочерних процессов. Взаимодействие между процессами осуществляется через системные сигналы/события и/или через отображаемые файлы (memory-mapped files).
Необходимо обрабатывать системные ошибки, которые могут возникнуть в результате работы.

\subsection*{Цель работы}
Изучение механизмов межпроцессного взаимодействия через отображаемые файлы (memory-mapped files) и синхронизации процессов с использованием семафоров.

\subsection*{Задание}
Реализовать программу, состоящую из родительского и двух дочерних процессов, взаимодействующих через отображаемый файл (memory-mapped file). 

Родительский процесс считывает строку от пользователя и записывает её в отображаемый файл. Затем создаются два дочерних процесса, которые последовательно обрабатывают данные в этом файле:

\begin{itemize}
    \item \textbf{Первый дочерний процесс} преобразует все буквы строки в верхний регистр
    \item \textbf{Второй дочерний процесс} заменяет все пробелы на символы подчеркивания
\end{itemize}

После завершения работы обоих дочерних процессов родительский процесс выводит финальный результат.

Для синхронизации доступа к общему файлу используются именованные семафоры POSIX.

\subsection*{Вариант}11