\section{Результаты тестирования}

\subsection{Пример работы программы}

Программа была скомпилирована и протестирована на ОС Linux. Ниже приведен пример её работы:

\begin{verbatim}
$ ./parent
Enter a string: hello world from test
Final output: HELLO_WORLD_FROM_TEST
\end{verbatim}

\subsection{Пошаговое описание работы}

\begin{enumerate}
    \item \textbf{Инициализация:} Родительский процесс создает разделяемую память и семафоры, создает и отображает файл \texttt{data\_to\_process.txt}
    
    \item \textbf{Ввод данных:} Пользователь вводит строку "hello world from test"
    
    \item \textbf{Создание дочерних процессов:} Родительский процесс создает два дочерних процесса с помощью \texttt{fork()} и \texttt{execl()}
    
    \item \textbf{Запуск первого процесса:} Родительский процесс освобождает \texttt{sem\_child}, разрешая первому дочернему процессу начать работу
    
    \item \textbf{Обработка первым процессом:} Первый дочерний процесс (\texttt{child1}):
    \begin{itemize}
        \item Захватывает семафоры \texttt{sem\_child} и \texttt{sem\_file}
        \item Отображает файл в память
        \item Преобразует строку в верхний регистр: "hello world from test" $\rightarrow$ "HELLO WORLD FROM TEST"
        \item Освобождает семафоры и сигнализирует второму процессу
    \end{itemize}
    
    \item \textbf{Обработка вторым процессом:} Второй дочерний процесс (\texttt{child2}):
    \begin{itemize}
        \item Захватывает семафоры \texttt{sem\_child} и \texttt{sem\_file}
        \item Отображает файл в память
        \item Заменяет пробелы на подчеркивания: "HELLO WORLD FROM TEST" $\rightarrow$ "HELLO\_WORLD\_FROM\_TEST"
        \item Освобождает семафоры и сигнализирует родительскому процессу
    \end{itemize}
    
    \item \textbf{Вывод результата:} Родительский процесс выводит итоговый результат обработки
    
    \item \textbf{Очистка ресурсов:} Родительский процесс удаляет файл, семафоры и разделяемую память
\end{enumerate}

\subsection{Тестовые примеры}

\subsubsection{Тест 1: Простая строка}
\begin{verbatim}
Вход: "hello world"
Выход: "HELLO_WORLD"
\end{verbatim}

\subsubsection{Тест 2: Строка с несколькими пробелами}
\begin{verbatim}
Вход: "this   is   a   test"
Выход: "THIS___IS___A___TEST"
\end{verbatim}

\subsubsection{Тест 3: Смешанный регистр}
\begin{verbatim}
Вход: "HeLLo WoRLd"
Выход: "HELLO_WORLD"
\end{verbatim}

\subsubsection{Тест 4: Строка с цифрами и символами}
\begin{verbatim}
Вход: "test123 program!"
Выход: "TEST123_PROGRAM!"
\end{verbatim}

\subsection{Анализ использованных системных вызовов}

В программе используются следующие системные вызовы POSIX:

\textbf{Работа с разделяемой памятью:}
\begin{itemize}
    \item \texttt{shm\_open()} -- создание/открытие объекта разделяемой памяти
    \item \texttt{shm\_unlink()} -- удаление объекта разделяемой памяти
    \item \texttt{ftruncate()} -- установка размера объекта
    \item \texttt{mmap()} -- отображение памяти в адресное пространство процесса
    \item \texttt{munmap()} -- отмена отображения памяти
\end{itemize}

\textbf{Работа с семафорами:}
\begin{itemize}
    \item \texttt{sem\_open()} -- создание/открытие именованного семафора
    \item \texttt{sem\_close()} -- закрытие семафора
    \item \texttt{sem\_unlink()} -- удаление именованного семафора
    \item \texttt{sem\_wait()} -- блокировка на семафоре
    \item \texttt{sem\_post()} -- разблокировка семафора
\end{itemize}

\textbf{Работа с процессами:}
\begin{itemize}
    \item \texttt{fork()} -- создание дочернего процесса
    \item \texttt{execl()} -- запуск новой программы в процессе
    \item \texttt{waitpid()} -- ожидание завершения дочернего процесса
\end{itemize}

\textbf{Работа с файлами:}
\begin{itemize}
    \item \texttt{open()} -- открытие файла
    \item \texttt{close()} -- закрытие файлового дескриптора
    \item \texttt{unlink()} -- удаление файла
    \item \texttt{read()} -- чтение из stdin
    \item \texttt{write()} -- запись в stdout
\end{itemize}

\textbf{Системные вызовы, используемые в программе:}
\begin{itemize}
    \item \texttt{shm\_open()}, \texttt{shm\_unlink()} - создание/удаление объекта разделяемой памяти "/shared"
    \item \texttt{mmap()}, \texttt{munmap()} - отображение/отмена отображения памяти и файлов
    \item \texttt{sem\_open()}, \texttt{sem\_close()}, \texttt{sem\_unlink()} - работа с именованными семафорами
    \item \texttt{sem\_wait()}, \texttt{sem\_post()} - операции синхронизации
    \item \texttt{fork()} - создание дочернего процесса
    \item \texttt{execl()} - замена образа процесса
    \item \texttt{open()}, \texttt{close()}, \texttt{ftruncate()} - работа с файлом data\_to\_process.txt
    \item \texttt{waitpid()} - ожидание завершения дочерних процессов
    \item \texttt{read()}, \texttt{write()} - ввод/вывод данных
    \item \texttt{unlink()} - удаление файла
\end{itemize}

\subsection{Особенности синхронизации}

Программа демонстрирует корректную работу механизма синхронизации с помощью семафоров:

\begin{itemize}
    \item Семафор \texttt{sem\_file} (инициализирован единицей) работает как мьютекс, предотвращая одновременный доступ к файлу
    
    \item Семафоры \texttt{sem\_child} и \texttt{sem\_parent} (инициализированы нулем) обеспечивают правильный порядок выполнения: сначала child1, затем child2, затем parent
    
    \item Использование memory-mapped файла позволяет дочерним процессам видеть изменения, сделанные друг другом, в реальном времени
\end{itemize}