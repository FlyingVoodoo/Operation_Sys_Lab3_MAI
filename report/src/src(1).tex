\section{Архитектура программы}

\subsection{Общая структура}
Программа реализует многопроцессную архитектуру с использованием memory-mapped файлов для межпроцессного взаимодействия. Основные компоненты системы:

\begin{itemize}
    \item \textbf{parent.cpp} - родительский процесс, управляет созданием дочерних процессов и координирует их работу
    \item \textbf{child1.cpp} - первый дочерний процесс, преобразует строку в верхний регистр
    \item \textbf{child2.cpp} - второй дочерний процесс, заменяет пробелы на подчеркивания
    \item \textbf{os.cpp/os.hpp} - библиотека с функциями для работы с процессами, семафорами и отображаемыми файлами
    \item \textbf{shared\_data.hpp} - структура данных для разделяемой памяти
\end{itemize}

\subsection{Механизм взаимодействия процессов}

\subsubsection{Memory-mapped файл}
Для обмена данными между процессами используется файл \texttt{data\_to\_process.txt}, который отображается в адресное пространство всех процессов с помощью системного вызова \texttt{mmap()}. Это позволяет всем процессам работать с одними и теми же данными в памяти.

\subsubsection{Синхронизация с помощью семафоров}
Для координации доступа к отображаемому файлу и правильного порядка выполнения используются три именованных POSIX-семафора:

\begin{itemize}
    \item \textbf{sem\_child} (начальное значение: 0) - сигнализирует дочерним процессам о готовности данных и координирует их последовательную работу
    \item \textbf{sem\_parent} (начальное значение: 0) - сигнализирует родительскому процессу о завершении обработки
    \item \textbf{sem\_file} (начальное значение: 1) - мьютекс для защиты доступа к файлу от одновременной записи
\end{itemize}

\subsection{Алгоритм работы программы}

\subsubsection{Родительский процесс (parent.cpp)}
\begin{enumerate}
    \item Создает разделяемую память для управляющих данных с помощью \texttt{shm\_open()} и инициализирует семафоры
    \item Создает и отображает файл \texttt{data\_to\_process.txt} для обмена данными
    \item Считывает строку от пользователя и записывает её в отображаемый файл
    \item Создает два дочерних процесса с помощью \texttt{fork()} и \texttt{execl()}
    \item Освобождает семафор \texttt{sem\_child}, разрешая первому дочернему процессу начать работу
    \item Ожидает сигнала от второго дочернего процесса на семафоре \texttt{sem\_parent}
    \item Выводит итоговый результат и очищает ресурсы
\end{enumerate}

\subsubsection{Первый дочерний процесс (child1.cpp)}
\begin{enumerate}
    \item Открывает разделяемую память и получает доступ к семафорам
    \item Ожидает разрешения на семафоре \texttt{sem\_child}
    \item Захватывает мьютекс \texttt{sem\_file} для эксклюзивного доступа к файлу
    \item Отображает файл в память и преобразует все символы в верхний регистр
    \item Освобождает мьютекс \texttt{sem\_file}
    \item Сигнализирует второму дочернему процессу через \texttt{sem\_child}
    \item Завершает работу
\end{enumerate}

\subsubsection{Второй дочерний процесс (child2.cpp)}
\begin{enumerate}
    \item Открывает разделяемую память и получает доступ к семафорам
    \item Ожидает сигнала от первого дочернего процесса на семафоре \texttt{sem\_child}
    \item Захватывает мьютекс \texttt{sem\_file}
    \item Отображает файл и заменяет все пробелы на символы подчеркивания
    \item Освобождает мьютекс \texttt{sem\_file}
    \item Сигнализирует родительскому процессу через \texttt{sem\_parent}
    \item Завершает работу
\end{enumerate}

\subsection{Использованные системные вызовы}

\begin{itemize}
    \item \texttt{shm\_open/shm\_unlink} - создание/удаление объекта разделяемой памяти POSIX
    \item \texttt{mmap/munmap} - отображение файла или памяти в адресное пространство процесса
    \item \texttt{sem\_open/sem\_close/sem\_unlink} - работа с именованными семафорами POSIX
    \item \texttt{sem\_wait/sem\_post} - операции ожидания и освобождения семафора
    \item \texttt{fork} - создание дочернего процесса
    \item \texttt{execl} - замена образа текущего процесса новой программой
    \item \texttt{waitpid} - ожидание завершения дочернего процесса
    \item \texttt{open/close} - открытие/закрытие файла
    \item \texttt{ftruncate} - установка размера файла
\end{itemize}